\documentclass[11pt,a4paper]{article}
\usepackage[utf8]{inputenc}
\usepackage[T1]{fontenc}
\usepackage[ngerman]{babel}
\usepackage{amsmath}
\usepackage{amsfonts}
\usepackage{amssymb}
\usepackage{graphicx}
\usepackage{packets}
\author{Lars Döpper}
\date{\today}
\title{441 Computerphysik - Hausaufgabe 3}
\begin{document}
	\maketitle
\section*{H.9: Grundzustand für (m,n)-Potentiale}
In dieser Hausaufgabe beschäftigen wir uns mit allgemeinen Potentialen der Form:
\begin{equation}
V_{m,n}(r) = V_0\left( \left(\frac{R}{r}\right)^{m} -\frac{m}{n}\left(\frac{R}{r}\right)^{n} \right)\frac{n}{m-n}
\end{equation}
Mit den Einschränkungen $V_0>0$ und $m>n$. Ziel dieser Hausaufgabe ist  die Bestimmung der Grundzustandsenergie des Lennard-Jones-Potentials mit $m=12 \; \&\; n=6$. Dazu gehen wir wieder von der zeitunabhängigen Schrödingergleichung aus:
\begin{equation}\label{eq:schrödinger}
	-\frac{\hbar^2}{2M}\Delta\phi(\vec{x}) + V(\vec{x})\phi(\vec{x}) = E\phi(\vec{x})
\end{equation}
Und setzen in diese dann das (m,n)-Potential ein und suchen nach dem ersten Eigenwert dieses Problems.
Wir interessieren uns allerdings nur für die sog. s-Wellen, also für die Wellen mit Drehimpulsquantenzahl $l=0$. Somit ist das Potential nicht mehr Richtungsabhängig und wir können das Potential im  Impulsraum schreiben als:
\begin{equation}
	V^{l=0}(p, p')=\frac{(4\pi)^2}{(2\pi)^3}\int_{0}^{\infty}r^2drV(r)\frac{\sin(pr)}{pr}\frac{\sin(p'r)}{p'r}
\end{equation}
Für allgemeine (m,n)-Potentiale müssen wir dieses Integral zumeist numerisch lösen. In dieser hausaufgabe verwenden wir dafür die Integrationsmethode nach Gauß und Legendre.
\subsection*{Das  (2,1)-Potential}
Um unsere Methoden zu entwickeln und auf ihre Richtigkeit zu prüfen, untersuchen wir zunächst das  (2,1)-Potential.  Dieses lautet dann:
\begin{equation}\label{eq:21pot}
	V(r) = V_0\left(\left(\frac{R}{r}\right)^2 -2\left(\frac{R}{r}\right)\right)
\end{equation}
\end{document}