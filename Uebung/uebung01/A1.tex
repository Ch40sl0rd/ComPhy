%%%%%%%%%%%%%%%%%%%%%%%%%%%%%%%%%%%%%%%%%%%%%%
%
% compile with
%
%  pdflatex A1.tex
%  pdflatex A1.tex
%
%%%%%%%%%%%%%%%%%%%%%%%%%%%%%%%%%%%%%%%%%%%%%%
\documentclass[11pt]{article}

\setlength{\parindent}{0cm}
\usepackage[english]{babel}
\usepackage[a4paper]{geometry}

\usepackage[english]{babel}
\usepackage[a4paper]{geometry}
\usepackage{amsmath}
\usepackage{amssymb}
\usepackage{graphicx}
\usepackage{color}
\usepackage{hyperref}

\begin{document}

\title{
Beispiel \LaTeX-Dokument f\"ur A1
}

\author{
  Marcus P
}

\date{\today}
\maketitle

\section{I/O und Funktionsanwendung f\"ur einen bin\"aren Datensatz}

Wir betrachten den Datensatz ``dataset2'' unter \url{link-to-dataset2}. In Bild \ref{fig:1} zeigen wir die Originaldaten auf der x-Achse und
\begin{align}
  y &= f(x) =\sin\left( x \right)
  \label{eq:1}
\end{align}
auf der y-Achse.
\begin{figure}[htpb]
  \centering
  \includegraphics[width=0.7\textwidth]{./dataset2_sin.eps}
  \caption{$\sin$-Funktion angewendet auf die Daten dataset2.}
  \label{fig:1}
\end{figure}

\end{document}
